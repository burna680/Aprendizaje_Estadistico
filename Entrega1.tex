% Options for packages loaded elsewhere
\PassOptionsToPackage{unicode}{hyperref}
\PassOptionsToPackage{hyphens}{url}
%
\documentclass[
]{article}
\usepackage{amsmath,amssymb}
\usepackage{lmodern}
\usepackage{iftex}
\ifPDFTeX
  \usepackage[T1]{fontenc}
  \usepackage[utf8]{inputenc}
  \usepackage{textcomp} % provide euro and other symbols
\else % if luatex or xetex
  \usepackage{unicode-math}
  \defaultfontfeatures{Scale=MatchLowercase}
  \defaultfontfeatures[\rmfamily]{Ligatures=TeX,Scale=1}
\fi
% Use upquote if available, for straight quotes in verbatim environments
\IfFileExists{upquote.sty}{\usepackage{upquote}}{}
\IfFileExists{microtype.sty}{% use microtype if available
  \usepackage[]{microtype}
  \UseMicrotypeSet[protrusion]{basicmath} % disable protrusion for tt fonts
}{}
\makeatletter
\@ifundefined{KOMAClassName}{% if non-KOMA class
  \IfFileExists{parskip.sty}{%
    \usepackage{parskip}
  }{% else
    \setlength{\parindent}{0pt}
    \setlength{\parskip}{6pt plus 2pt minus 1pt}}
}{% if KOMA class
  \KOMAoptions{parskip=half}}
\makeatother
\usepackage{xcolor}
\IfFileExists{xurl.sty}{\usepackage{xurl}}{} % add URL line breaks if available
\IfFileExists{bookmark.sty}{\usepackage{bookmark}}{\usepackage{hyperref}}
\hypersetup{
  pdftitle={Entrega1},
  hidelinks,
  pdfcreator={LaTeX via pandoc}}
\urlstyle{same} % disable monospaced font for URLs
\usepackage[margin=1in]{geometry}
\usepackage{color}
\usepackage{fancyvrb}
\newcommand{\VerbBar}{|}
\newcommand{\VERB}{\Verb[commandchars=\\\{\}]}
\DefineVerbatimEnvironment{Highlighting}{Verbatim}{commandchars=\\\{\}}
% Add ',fontsize=\small' for more characters per line
\usepackage{framed}
\definecolor{shadecolor}{RGB}{248,248,248}
\newenvironment{Shaded}{\begin{snugshade}}{\end{snugshade}}
\newcommand{\AlertTok}[1]{\textcolor[rgb]{0.94,0.16,0.16}{#1}}
\newcommand{\AnnotationTok}[1]{\textcolor[rgb]{0.56,0.35,0.01}{\textbf{\textit{#1}}}}
\newcommand{\AttributeTok}[1]{\textcolor[rgb]{0.77,0.63,0.00}{#1}}
\newcommand{\BaseNTok}[1]{\textcolor[rgb]{0.00,0.00,0.81}{#1}}
\newcommand{\BuiltInTok}[1]{#1}
\newcommand{\CharTok}[1]{\textcolor[rgb]{0.31,0.60,0.02}{#1}}
\newcommand{\CommentTok}[1]{\textcolor[rgb]{0.56,0.35,0.01}{\textit{#1}}}
\newcommand{\CommentVarTok}[1]{\textcolor[rgb]{0.56,0.35,0.01}{\textbf{\textit{#1}}}}
\newcommand{\ConstantTok}[1]{\textcolor[rgb]{0.00,0.00,0.00}{#1}}
\newcommand{\ControlFlowTok}[1]{\textcolor[rgb]{0.13,0.29,0.53}{\textbf{#1}}}
\newcommand{\DataTypeTok}[1]{\textcolor[rgb]{0.13,0.29,0.53}{#1}}
\newcommand{\DecValTok}[1]{\textcolor[rgb]{0.00,0.00,0.81}{#1}}
\newcommand{\DocumentationTok}[1]{\textcolor[rgb]{0.56,0.35,0.01}{\textbf{\textit{#1}}}}
\newcommand{\ErrorTok}[1]{\textcolor[rgb]{0.64,0.00,0.00}{\textbf{#1}}}
\newcommand{\ExtensionTok}[1]{#1}
\newcommand{\FloatTok}[1]{\textcolor[rgb]{0.00,0.00,0.81}{#1}}
\newcommand{\FunctionTok}[1]{\textcolor[rgb]{0.00,0.00,0.00}{#1}}
\newcommand{\ImportTok}[1]{#1}
\newcommand{\InformationTok}[1]{\textcolor[rgb]{0.56,0.35,0.01}{\textbf{\textit{#1}}}}
\newcommand{\KeywordTok}[1]{\textcolor[rgb]{0.13,0.29,0.53}{\textbf{#1}}}
\newcommand{\NormalTok}[1]{#1}
\newcommand{\OperatorTok}[1]{\textcolor[rgb]{0.81,0.36,0.00}{\textbf{#1}}}
\newcommand{\OtherTok}[1]{\textcolor[rgb]{0.56,0.35,0.01}{#1}}
\newcommand{\PreprocessorTok}[1]{\textcolor[rgb]{0.56,0.35,0.01}{\textit{#1}}}
\newcommand{\RegionMarkerTok}[1]{#1}
\newcommand{\SpecialCharTok}[1]{\textcolor[rgb]{0.00,0.00,0.00}{#1}}
\newcommand{\SpecialStringTok}[1]{\textcolor[rgb]{0.31,0.60,0.02}{#1}}
\newcommand{\StringTok}[1]{\textcolor[rgb]{0.31,0.60,0.02}{#1}}
\newcommand{\VariableTok}[1]{\textcolor[rgb]{0.00,0.00,0.00}{#1}}
\newcommand{\VerbatimStringTok}[1]{\textcolor[rgb]{0.31,0.60,0.02}{#1}}
\newcommand{\WarningTok}[1]{\textcolor[rgb]{0.56,0.35,0.01}{\textbf{\textit{#1}}}}
\usepackage{graphicx}
\makeatletter
\def\maxwidth{\ifdim\Gin@nat@width>\linewidth\linewidth\else\Gin@nat@width\fi}
\def\maxheight{\ifdim\Gin@nat@height>\textheight\textheight\else\Gin@nat@height\fi}
\makeatother
% Scale images if necessary, so that they will not overflow the page
% margins by default, and it is still possible to overwrite the defaults
% using explicit options in \includegraphics[width, height, ...]{}
\setkeys{Gin}{width=\maxwidth,height=\maxheight,keepaspectratio}
% Set default figure placement to htbp
\makeatletter
\def\fps@figure{htbp}
\makeatother
\setlength{\emergencystretch}{3em} % prevent overfull lines
\providecommand{\tightlist}{%
  \setlength{\itemsep}{0pt}\setlength{\parskip}{0pt}}
\setcounter{secnumdepth}{-\maxdimen} % remove section numbering
\ifLuaTeX
  \usepackage{selnolig}  % disable illegal ligatures
\fi

\title{Entrega1}
\author{}
\date{\vspace{-2.5em}2022-05-26}

\begin{document}
\maketitle

Todos los pacientes son de sexo femenino mayores de 21 años de herencia
Pima. Para este primer estudio, lo que buscaremos es encontrar
relaciones entre otras de las variables contenidas en el dataset.

\hypertarget{punto-1}{%
\subsection{Punto 1}\label{punto-1}}

Se analizaron los datos y se llega a la conclusión de que todas las
columnas se encuentran bien definidas.

La columna X parecería ser un contador ya que es siempre ascendiente
aunque faltan algunos numeros, podría ser un identificador de muestras.
Por lo tanto no lo tomamos en cuenta para los análisis siguientes.

Revisando los valores se nota que algunas filas tienen algunos valores
en 0 que no tendrían mucho sentido en este contexto medicinal. Por
ejemplo no se le encuentra mucho sentido que la presión sanguínea sea 0.
Por lo tanto se comienza una exploración de datos para ver que
porcentaje de los datos tienen valores en 0. Igualmente este análisis se
deja para el punto 10 de selección de modelos.\\
\strut \\

\hypertarget{punto-2}{%
\subsection{\texorpdfstring{\textbf{Punto 2}}{Punto 2}}\label{punto-2}}

El modelo resulta

\(\Omega\): Y= X\(\beta\)+\(\epsilon\)

Cuyos supuestos son:

\begin{enumerate}
\def\labelenumi{\arabic{enumi}.}
\item
  Y \(\sim\) \(\mathcal{N}\)(X\(\beta\), \(\sigma^2I\)) , los valores de
  BMI

  Y \(\in\) \(\mathcal{R} ^{nx1}, n=539\) la cantidad de observaciones
\item
  \(E(Y|X=x)= \beta_0+\beta_1x_1+...+\beta_{p-1}x_{p-1}\))
\item
  \(Var(Y|X=x)\)= \(\sigma^2\)
\item
  \(Y_{1}, \ldots, Y_{n}\) son independientes

  Al definir los elementos\\
  \[
  Y=\left(y_{1}, \ldots, y_{n}\right)^{T}, \epsilon=\left(\epsilon_{1}, \ldots, \epsilon_{n}\right)^{T}, \beta=\left(\beta_{1}, \ldots, \beta_{p}\right)^{T}
  \]

  \(\beta\in\) \(\mathcal{R} ^{px1}\)

  X \(\in\) \(\mathcal{R} ^{nxp}, p= 8\) Es la matriz de diseño. La
  cantidad de variables del modelo es p.~Incluyen la cantidad de
  embarazos, el nivel de glucosa, etc.
  \[\mathbf{X}=\left[\begin{array}{cccc}1 & x_{11} & \ldots & x_{1(p-1)} \\1 & \cdot & \cdot & \cdot \\1 & \cdot & \cdot & \cdot \\1 & \cdot & \cdot & \cdot \\1 & x_{n 1} & \ldots & x_{n(p-1)}\end{array}\right]\]

  \hypertarget{punto-3}{%
  \subsection{Punto 3}\label{punto-3}}
\end{enumerate}

La siguiente tabla muestra las correlaciones entre los distintos pares
de variables del dataset. Se puede ver que en la última columna no se
muestran los resultados como en las otras ya que la variable Outcome es
una variable categórica.

\begin{center}\includegraphics{Entrega1_files/figure-latex/unnamed-chunk-3-1} \end{center}

\hypertarget{punto-4}{%
\subsection{Punto 4}\label{punto-4}}

Al ver la tabla podemos ver que existe una mayor correlación entre los
valores de la SkinThickness y los de BMI, por lo que si tuviéramos que
perder información eligiendo sólo una variable de las contenidas en el
dataset sería la SkinThickness ya que sería un modelo de regresión
lineal simple y la correlación da una noción de como explica una
variable a la otra.

\hypertarget{punto-5}{%
\subsection{Punto 5}\label{punto-5}}

\begin{verbatim}
## 
## Call:
## lm(formula = data$BMI ~ vars, data = data)
## 
## Residuals:
##      Min       1Q   Median       3Q      Max 
## -23.4933  -3.3207  -0.6014   3.1331  22.2634 
## 
## Coefficients:
##                               Estimate Std. Error t value Pr(>|t|)    
## (Intercept)                  13.977191   1.377092  10.150  < 2e-16 ***
## varsGlucose                  -0.002452   0.008304  -0.295   0.7679    
## varsPregnancies              -0.083276   0.085507  -0.974   0.3305    
## varsBloodPressure             0.092145   0.017893   5.150 3.68e-07 ***
## varsSkinThickness             0.378576   0.021711  17.437  < 2e-16 ***
## varsInsulin                   0.004816   0.001987   2.424   0.0157 *  
## varsDiabetesPedigreeFunction  0.897675   0.650422   1.380   0.1681    
## varsAge                      -0.063677   0.028000  -2.274   0.0234 *  
## varsOutcome                   2.184222   0.546829   3.994 7.40e-05 ***
## ---
## Signif. codes:  0 '***' 0.001 '**' 0.01 '*' 0.05 '.' 0.1 ' ' 1
## 
## Residual standard error: 4.98 on 530 degrees of freedom
## Multiple R-squared:  0.4825, Adjusted R-squared:  0.4747 
## F-statistic: 61.78 on 8 and 530 DF,  p-value: < 2.2e-16
\end{verbatim}

En la anterior tabla se puede ver el resultado de la regresión lineal
múltiple.

En la parte de coeficientes cada fila expresa medidas de cada
coeficiente de los parámetros estimados (\(\beta_i\)). La primer columna
es el valor de la estimación, y después las otras tres columnas son
tests de hipótesis para cada estimación de la siguiente forma:

\[
H_{0}: \beta_{i}=0 \quad V s . \quad H_{1}: \beta_{i} \neq 0
\]

La regla de decisión será

\[
\varphi(\underline{X})= \begin{cases}1 & \text { si } \quad|T|>k_{\alpha} \\ 0 & \text { en otro caso }\end{cases}
\]

Donde

\[
T=\frac{\hat{\beta_i}}{S \sqrt{d_{i i}}}, T ∼ t_{n - p}
\]

Donde S es la estimación (insesgada) del desvío estándar del estimador.

Y \(d_{ii}\) es el elemento de la matriz
\(D=\left(\mathbf{X}^{T} \mathbf{X}\right)^{-1}\)

\(k_{\alpha}\) es una constante que depende de \(\alpha\) que es el
nivel de significación del test.

Por último el p-valor se calcula de la siguiente manera:

\[
p-\text { valor }=2 \mathbf{P}\left(T \geq\left|T_{o b s}\right|\right)
\]

Entonces la columna de Std. Error es la multiplicación de \(s\) y
\(\sqrt{d_{ii}}\), t value es el valor obtenido del estadístico T, osea:
\(\frac{\hat{\beta_i}}{s \sqrt{d_{i i}}}\)

Y la última columna es el p valor obtenido para ese \(t\), osea:
\(2 \mathbf{P}\left(T \geq\left|t\right|\right)\)

Como ejemplo se toma la variable SkinThickness y se calcula a ``mano''
los valores:

\[S^{2}=\frac{\|Y-\hat{Y}\|^{2}}{n-p}\]

\begin{Shaded}
\begin{Highlighting}[]
\NormalTok{Y\_hat }\OtherTok{=} \FunctionTok{predict}\NormalTok{(lin\_reg)}
\NormalTok{s }\OtherTok{=} \FunctionTok{sqrt}\NormalTok{(}\FunctionTok{norm}\NormalTok{(}\FunctionTok{as.matrix}\NormalTok{(data}\SpecialCharTok{$}\NormalTok{BMI }\SpecialCharTok{{-}}\NormalTok{ Y\_hat), }\AttributeTok{type =} \StringTok{"2"}\NormalTok{)}\SpecialCharTok{**}\DecValTok{2} \SpecialCharTok{/}\NormalTok{ (}\DecValTok{539} \SpecialCharTok{{-}} \DecValTok{8}\NormalTok{))}
\NormalTok{s}
\end{Highlighting}
\end{Shaded}

\begin{verbatim}
## [1] 4.975789
\end{verbatim}

Que es el mismo que aparece en la tabla como Residual Standar Error.

\begin{Shaded}
\begin{Highlighting}[]
\NormalTok{X }\OtherTok{=} \FunctionTok{cbind}\NormalTok{(}\FunctionTok{rep}\NormalTok{(}\FunctionTok{c}\NormalTok{(}\DecValTok{1}\NormalTok{), }\DecValTok{539}\NormalTok{), vars) }\CommentTok{\# vars contiene las columnas en el orden igual al que se hizo la regresion lineal}
\NormalTok{D }\OtherTok{=} \FunctionTok{solve}\NormalTok{(}\FunctionTok{t}\NormalTok{(X) }\SpecialCharTok{\%*\%}\NormalTok{ X)}
\NormalTok{d44 }\OtherTok{=}\NormalTok{ D[}\DecValTok{5}\NormalTok{, }\DecValTok{5}\NormalTok{] }\CommentTok{\# ya que el 0,0 sería el del intercept}
\NormalTok{std\_err\_b4 }\OtherTok{=}\NormalTok{ (}\FunctionTok{sqrt}\NormalTok{(d44) }\SpecialCharTok{*}\NormalTok{ s)}
\NormalTok{std\_err\_b4}
\end{Highlighting}
\end{Shaded}

\begin{verbatim}
## [1] 0.02169034
\end{verbatim}

Que es el mismo valor que se obtiene en la tabla como Std. Error.

\begin{Shaded}
\begin{Highlighting}[]
\NormalTok{t\_b4 }\OtherTok{=} \FunctionTok{unname}\NormalTok{(lin\_reg}\SpecialCharTok{$}\NormalTok{coefficients[}\StringTok{\textquotesingle{}varsSkinThickness\textquotesingle{}}\NormalTok{]}\SpecialCharTok{/}\NormalTok{std\_err\_b4)}
\NormalTok{t\_b4}
\end{Highlighting}
\end{Shaded}

\begin{verbatim}
## [1] 17.45366
\end{verbatim}

Que es el valor que se obtiene de la tabla como t value

Por último se puede calcular el p valor usando la función de
distribución de la t de Student

\begin{Shaded}
\begin{Highlighting}[]
\DecValTok{2}\SpecialCharTok{*}\NormalTok{(}\DecValTok{1} \SpecialCharTok{{-}} \FunctionTok{pt}\NormalTok{(t\_b4, }\DecValTok{539} \SpecialCharTok{{-}} \DecValTok{8}\NormalTok{))}
\end{Highlighting}
\end{Shaded}

\begin{verbatim}
## [1] 0
\end{verbatim}

Ahora mirando todos los resultados se puede decidir que valores son
significativos de la estimación y se decide que las variables con un
valor significativo (\(p-valor < 0.05\)) del test son:

\begin{itemize}
\item
  Intercept (\(\beta_0\))
\item
  BloodPressure
\item
  SkinThickness
\item
  Insulin
\item
  DiabetesPedigreeFunction
\end{itemize}

\hypertarget{punto-6}{%
\subsection{Punto 6}\label{punto-6}}

\[
R^{2}=\frac{\|\hat{Y}-\bar{Y}\|^{2}}{\|Y-\bar{Y}\|^{2}}=\frac{S C R}{S C T}= 0.4669
\]

Donde \(\hat Y\) es la estimación de \(Y\) e \(\bar Y\) es la media. El
\(R^2\) da una medida de la capacidad de ajuste del modelo, es decir que
da una noción de cuanta variabilidad de \(Y\) queda explicada por el
modelo. Cuando el valor se encuentra más cercano a 1 quiere decir que el
modelo explica mejor la variabilidad.

A medida que se agregan variables al modelo este valor siempre crece, a
pesar de que estas nuevas variable no aporten a la estimación. Por eso
es que cuando se quiere comparar dos modelos es preferible usar el
coeficiente de determinación ajustado como:

\[
R_{a}^{2}=1-\frac{n-1}{n-p} \frac{\|Y-\hat{Y}\|^{2}}{\|Y-\bar{Y}\|^{2}}=1-\left(1-R^{2}\right) \frac{n}{n-p}
\]

\hypertarget{punto-7}{%
\subsection{Punto 7}\label{punto-7}}

Si se quiere analizar si la regresión es significativa se puede hacer el
siguiente test

\[H_{0}: \beta_{1}=\cdots=\beta_{p-1}=0 \quad \text { Vs. } \quad H_{1}: \text { Algun } \quad \beta_{i} \neq 0, \quad i=1, \ldots, p-1\]

\[
\varphi(\underline{X})= \begin{cases}1 & \text { si } \quad F>\mathcal{F}_{p-1, n-p, 1-\alpha} \\ 0 & \text { en otro caso }\end{cases}
\]

Donde:

\[
F=\frac{(C \hat{\beta})^{T}\left(C\left(\mathbf{X}^{T} \mathbf{X}\right)^{-1} C^{T}\right)^{-1}(C \hat{\beta})}{(p - 1) S^{2}} 
\]

Con:

\[
\mathbf{C}=\left[\begin{array}{ccc}0 & 1 \ldots & 0 \\0 & . & . \\. & . & \cdot \\. & \cdot & \cdot \\0 & \ldots & 1\end{array}\right] ; C\in \mathcal{R}^{(p-1)\times p}
\]

Que representa la manera que está armado el test
(\(\beta_{1}=0 ; \beta_{2}=0 ...\))

En este caso el p-valor se calcula como:

\[
p-\text { valor }=\mathbf{P}\left(F \geq F_{o b s}\right), \quad F \sim \mathcal{F}_{p - 1, n-p}
\]

Por lo tanto en este caso el p-valor está tan cercano a 0 que significa
que la regresión es significativa.

\hypertarget{punto-8}{%
\subsection{Punto 8}\label{punto-8}}

En el caso de querer obtener una estimación de la esperanza para el BMI,
para una mujer que tuvo 2 embarazos, concentración de glucosa de 100,
presión sanguínea de 70, piel de triceps de 20, sin diabetes, un valor
de la función pedigree de 0.24 y 30 años usamos el modelo creado en el
punto 5, cuyo valor resulta:

\begin{Shaded}
\begin{Highlighting}[]
\FunctionTok{predict}\NormalTok{(modelo, newdata)}
\end{Highlighting}
\end{Shaded}

\begin{verbatim}
##        1 
## 28.61348
\end{verbatim}

\hypertarget{punto-9}{%
\subsection{Punto 9}\label{punto-9}}

Dado el modelo descrito en el punto 2, se quiere obtener un intervalo de
confianza para la estimación de la esperanza de una nueva observación
\(\mathbf{E}\left(Y_{0}\right)=x_{0}^{T} \beta\)\\
\strut \\
Para ello tenemos en cuenta el desarrollo del método del pivote
utilizando
\[T=\frac{\hat{Y}_{0}-x_{0}^{T} \beta}{S \sqrt{x_{0}^{T}\left(\mathbf{X}^{T} \mathbf{X}\right)^{-1} x_{0}}} \sim t_{n-p}\]

que resulta en un intervalo de confianza

\[ I C_{1-0.05}=\left[\hat{y_{0}}-t_{n-p, 1-\alpha / 2} S \sqrt{x_{0}^{T}\left(\mathbf{X}^{T} \mathbf{X}\right)^{-1} x_{0}} ; \hat{y}_{0}+t_{n-p, 1-\alpha / 2} S \sqrt{x_{0}^{T}\left(\mathbf{X}^{T} \mathbf{X}\right)^{-1} x_{0}}\right] \]

Puesto que el nivel que necesitamos es 0,95, el resultado es

\begin{Shaded}
\begin{Highlighting}[]
\FunctionTok{predict}\NormalTok{(modelo, }\AttributeTok{newdata =}\NormalTok{ newdata, }\AttributeTok{interval =} \StringTok{"confidence"}\NormalTok{) }\CommentTok{\#Intervalo de confianza de nivel 0.95 (por default) }
\end{Highlighting}
\end{Shaded}

\begin{verbatim}
##        fit      lwr      upr
## 1 28.61348 27.91714 29.30982
\end{verbatim}

Además se desea un intervalo de predicción para la variable
\[\hat{Y}_{0}=x_{0}^{T} \hat{\beta}\],
\[ \hat{Y}_{0} \sim \mathcal{N}\left(x_{0}^{T} \beta, \sigma^{2} x_{0}^{T}\left(\mathbf{X}^{T} \mathbf{X}\right)^{-1} x_{0}\right)\]

Nuevamente utilizando el método del pivote, resulta que el intervalo de
predicción es:
\[I C_{1-0.05}=\left[\hat{y}_{0}-t_{n-p, 1-\alpha / 2} S \sqrt{1+x_{0}^{T}\left(\mathbf{X}^{T} \mathbf{X}\right)^{-1} x_{0}} ; \hat{y}_{0}+t_{n-p, 1-\alpha / 2} S \sqrt{1+x_{0}^{T}\left(\mathbf{X}^{T} \mathbf{X}\right)^{-1} x_{0}}\right] \]

y el resultado:

\begin{Shaded}
\begin{Highlighting}[]
\FunctionTok{predict}\NormalTok{(modelo, }\AttributeTok{newdata =}\NormalTok{ newdata, }\AttributeTok{interval =} \StringTok{"prediction"}\NormalTok{) }\CommentTok{\#Intervalo de prediccion de nivel 0.95 (por default) }
\end{Highlighting}
\end{Shaded}

\begin{verbatim}
##        fit      lwr     upr
## 1 28.61348 18.62647 38.6005
\end{verbatim}

Vemos que el intervalo de predicción es mayor que el de confianza,
resultado esperado al tratar de obtener una predicción de un valor
exacto.

\hypertarget{punto-10}{%
\subsection{Punto 10}\label{punto-10}}

Para la selección de modelos se consideran todas las posibles
combinaciones de variables para poder encontrar el mejor de los modelos.
Para considerar cual es el mejor es necesario definir métricas.

Se utilizaran 3 métricas:

\begin{itemize}
\item
  \[
  R_{a}^{2}=1-\frac{n-1}{n-p} \frac{\|Y-\hat{Y}\|^{2}}{\|Y-\bar{Y}\|^{2}}
  \]
\item
  \[
  C p=\frac{\left\|Y-\hat{Y}_{p}\right\|^{2}}{S^{2}}+2 p-n, \quad \quad S^{2}=\frac{\left\|Y-\hat{Y}_{k}\right\|^{2}}{n-k}
  \]
\item
  \[
  \widehat{E C M}=\frac{1}{n} \sum_{i=1}^{n} r_{-i}^{2}=\frac{1}{n} \sum_{i=1}^{n} \frac{r^{2}}{\left(1-p_{i i}\right)^{2}}
  \]
\end{itemize}

La bondad de ajuste está dada por el \(R_{a}^{2}\), mientras que la
bondad de predicción está dada por el \(C_p\) . Otra métrica también
utilizada es el error cuadrático medio para verificar cuanto se aleja la
estimación de la medición, para este caso se usó el método de CV para
medir esta métrica.

\begin{center}\includegraphics{Entrega1_files/figure-latex/unnamed-chunk-13-1} \end{center}

Utilizando la función regsubsets con el metodo exhaustivo se obtuvieron
los mejores (se queda con el modelo con mayor F) modelos para la
cantidad de variables. En el gráfico de arriba se pueden ver como cambia
el \(R_{a}^{2}\) y el \(C_p\) a medida que cambian la cantidad de
variables. De aquí se podría considerar que el óptimo se obtiene con 5
variables, pero no hay mucha diferencia con los menores. Teniendo en
cuenta de que se prefiere usar menor cantidad de variables se podría
tomar el de 4 o 3.

\begin{verbatim}
## Subset selection object
## Call: stepwise(data)
## 8 Variables  (and intercept)
##                              Forced in Forced out
## varsGlucose                      FALSE      FALSE
## varsPregnancies                  FALSE      FALSE
## varsBloodPressure                FALSE      FALSE
## varsSkinThickness                FALSE      FALSE
## varsInsulin                      FALSE      FALSE
## varsDiabetesPedigreeFunction     FALSE      FALSE
## varsAge                          FALSE      FALSE
## varsOutcome                      FALSE      FALSE
## 1 subsets of each size up to 8
## Selection Algorithm: exhaustive
##          varsGlucose varsPregnancies varsBloodPressure varsSkinThickness
## 1  ( 1 ) " "         " "             " "               "*"              
## 2  ( 1 ) " "         " "             "*"               "*"              
## 3  ( 1 ) " "         " "             "*"               "*"              
## 4  ( 1 ) " "         " "             "*"               "*"              
## 5  ( 1 ) " "         " "             "*"               "*"              
## 6  ( 1 ) " "         " "             "*"               "*"              
## 7  ( 1 ) " "         "*"             "*"               "*"              
## 8  ( 1 ) "*"         "*"             "*"               "*"              
##          varsInsulin varsDiabetesPedigreeFunction varsAge varsOutcome
## 1  ( 1 ) " "         " "                          " "     " "        
## 2  ( 1 ) " "         " "                          " "     " "        
## 3  ( 1 ) " "         " "                          " "     "*"        
## 4  ( 1 ) " "         " "                          "*"     "*"        
## 5  ( 1 ) "*"         " "                          "*"     "*"        
## 6  ( 1 ) "*"         "*"                          "*"     "*"        
## 7  ( 1 ) "*"         "*"                          "*"     "*"        
## 8  ( 1 ) "*"         "*"                          "*"     "*"
\end{verbatim}

En esta tabla se puede ver cuales variables se utilizan para la cantidad
de variables.

Para calcular el error cuadratico medio en cambio se usa la función
bestglm ya que elije el mejor modelo (en este caso se le puede pedir que
el mejor sea el que tenga menor ECM) usando cross validation.

\begin{verbatim}
##    (Intercept) Glucose Pregnancies BloodPressure SkinThickness Insulin
## 0         TRUE   FALSE       FALSE         FALSE         FALSE   FALSE
## 1         TRUE   FALSE       FALSE         FALSE          TRUE   FALSE
## 2         TRUE   FALSE       FALSE          TRUE          TRUE   FALSE
## 3         TRUE   FALSE       FALSE          TRUE          TRUE   FALSE
## 4*        TRUE   FALSE       FALSE          TRUE          TRUE   FALSE
## 5         TRUE   FALSE       FALSE          TRUE          TRUE    TRUE
## 6         TRUE   FALSE       FALSE          TRUE          TRUE    TRUE
## 7         TRUE   FALSE        TRUE          TRUE          TRUE    TRUE
## 8         TRUE    TRUE        TRUE          TRUE          TRUE    TRUE
##    DiabetesPedigreeFunction   Age Outcome logLikelihood       CV
## 0                     FALSE FALSE   FALSE    -1038.3862 47.71661
## 1                     FALSE FALSE   FALSE     -891.4980 28.45337
## 2                     FALSE FALSE   FALSE     -880.9594 28.02482
## 3                     FALSE FALSE    TRUE     -872.7179 27.50207
## 4*                    FALSE  TRUE    TRUE     -866.2922 27.24904
## 5                     FALSE  TRUE    TRUE     -862.3907 27.29241
## 6                      TRUE  TRUE    TRUE     -861.3503 27.49842
## 7                      TRUE  TRUE    TRUE     -860.8852 28.04365
## 8                      TRUE  TRUE    TRUE     -860.8408 28.32924
\end{verbatim}

Este algoritmo encuentra que el mejor modelo (el que minimiza el ECM) es
el de 4 variables.

Comparando los dos resultados elegimos el de 5 variables (en este caso
dieron iguales) ya que para el segundo análisis no se encuentra muy
alejado del mínimo, y para el primer análisis se encuentra cerca del
óptimo.

El modelo entonces resulta con los siguientes estimadores:

\hypertarget{punto-11}{%
\subsection{Punto 11}\label{punto-11}}

Recordando los supuestos del modelo:

\begin{enumerate}
\def\labelenumi{\arabic{enumi}.}
\item
  . Los errores tienen media cero E(\(\epsilon\))=0
\item
  Homocedasticidad: los errores tienen todos la misma varianza\\
  \strut \\
  Var(\(\epsilon\))= \(\sigma^2\)
\item
  Los errores tienen distribución normal, son independientes entre sí y
  no están correlacionados con las variables de entrada
\end{enumerate}

Que también se pueden ver como

\begin{enumerate}
\def\labelenumi{\arabic{enumi}.}
\item
  Y \(\sim\) \(\mathcal{N}\)(X\(\beta\), \(\sigma^2I\))
\item
  \(E(Y|X=x)= \beta_0+\beta_1x_1+...+\beta_{p-1}x_{p-1}\)
\item
  \(Var(Y|X=x)\)= \(\sigma^2\)
\item
  \(Y_{1}, \ldots, Y_{n}\) son independientes
\end{enumerate}

Al tener un modelo multivariable es necesario analizar los residuos para
comprender si los supuestos son válidos en el modelo propuesto. En
principio, el mirar un gráfico de los residuos y reconocer patrones es
un indicio de que el modelo no es adecuado. Deberían ser puntos
distribuidos aleatoriamente con media cero.

\begin{Shaded}
\begin{Highlighting}[]
\FunctionTok{par}\NormalTok{(}\AttributeTok{mfrow =} \FunctionTok{c}\NormalTok{(}\DecValTok{2}\NormalTok{, }\DecValTok{2}\NormalTok{))}
\FunctionTok{plot}\NormalTok{(modelo)}
\end{Highlighting}
\end{Shaded}

\begin{center}\includegraphics{Entrega1_files/figure-latex/unnamed-chunk-17-1} \end{center}

\begin{Shaded}
\begin{Highlighting}[]
\CommentTok{\# }
\CommentTok{\# residuals\_std\textless{}{-}rstandard(modelo) \#t{-}residuals}
\CommentTok{\# }
\CommentTok{\# ggplot(data=data, aes(x=1:length(modelo$residuals), y=modelo$residuals)) + }
\CommentTok{\#   geom\_point(col="gray0", fill="darkorchid3", alpha = .9) + }
\CommentTok{\#   labs(title="Residuos", x="Indice", y="Valores") + xlim(c(0,length(modelo$residuals))) + ylim(c(min(modelo$residuals),max(modelo$residuals)))}
\end{Highlighting}
\end{Shaded}

En el gráfico superior izquierdo se puede verificar que la media es
nula, y el supuesto de linealidad es válido.

Además, sabemos que podemos comprar ,mediante el gráfico superior
derecho, la normalidad de los supuestos. Vemos que los residuos siguen
una distribución normal.

Luego, con el gráfico inferior izquierdo podemos comprobar la
homeocedasticidad, puesto que la varianza de los residuos es
relativamente constante.

Con el gráfico inferior derecho se verifica la presencia de outliers,
que son valores extremos dentro de nuestro set y que pueden afectar al
Residual Standar Error.

\hypertarget{punto-12}{%
\subsection{Punto 12}\label{punto-12}}

Como se había elegido la columna SkinThickness, se realiza el bootstrap
no paramétrico y como remuestra se toma la salida de la regresión
simple. Se toman 10000 muestras del estimador y se gráfica la densidad
usando nucleo gaussiano, se compara contra una normal y se calculan la
media y el desvío estandard.

\begin{Shaded}
\begin{Highlighting}[]
\FunctionTok{mean}\NormalTok{(estimations)}
\end{Highlighting}
\end{Shaded}

\begin{verbatim}
## [1] 0.4275675
\end{verbatim}

\begin{Shaded}
\begin{Highlighting}[]
\FunctionTok{var}\NormalTok{(estimations)}
\end{Highlighting}
\end{Shaded}

\begin{verbatim}
## [1] 0.001382514
\end{verbatim}

\begin{center}\includegraphics{Entrega1_files/figure-latex/unnamed-chunk-20-1} \end{center}

\begin{center}\includegraphics{Entrega1_files/figure-latex/unnamed-chunk-21-1} \end{center}

Se puede notar que es similar a una gaussiana, esto es útil ya que
indica que el valor no varía mucho.

\end{document}
